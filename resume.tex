\documentclass{resume} % Use the custom resume.cls style
\usepackage[T1]{fontenc}
\usepackage{fontawesome5}
\usepackage{titlesec}
\usepackage{charter}
\usepackage{lettrine}
\usepackage{amsmath}
\usepackage{graphicx}
\usepackage{wrapfig}
\usepackage{letltxmacro,xpatch}
\LetLtxMacro\oldLaTeX\LaTeX
\LetLtxMacro\oldTeX\TeX
\xpatchcmd{\oldLaTeX}{\TeX}{\oldTeX}{}{\ddt}

\usepackage{metalogo}

\setLaTeXa{\scshape a}
\setlogokern{La}{-.3em}
\setlogokern{aT}{-.1em}
\setlogokern{Te}{-.08em}
\setlogokern{eX}{-.1em}
\setlogodrop{0.33ex}
\graphicspath{ {images/} }

% Removed duplicate geometry package loading since it's already in resume.cls
% If you need to modify margins, do it in the resume.cls file instead

\newcommand{\tab}[1]{\hspace{.2667\textwidth}\rlap{#1}} 
\newcommand{\itab}[1]{\hspace{0em}\rlap{#1}}

\name{\textsc{\scalebox{1.25}{N}inad \scalebox{1.25}{N}aik}}

\address{
\faIcon{phone-square} 
+91 97693 31376 \\
\faIcon{envelope} \href{mailto:ninadnaik07@gmail.com}{ninadnaik07@gmail.com} \\ \faIcon{linkedin} \href{https://linkedin.com/in/ninadn}{/ninadn} \\ \faIcon{github} \href{https://github.com/ninadnaik10}{/ninadnaik10} \\
\faIcon{globe} \href{https://ninadnaik.xyz}{ninadnaik.xyz}
}

\begin{document}

%----------------------------------------------------------------------------------------
%	OBJECTIVE
%----------------------------------------------------------------------------------------

%\begin{rSection}{OBJECTIVE}

%{A Tech-Enthusiast who wants learn the prospect of technology and the internals of computing.}

%\end{rSection}
%----------------------------------------------------------------------------------------
%	EDUCATION SECTION
%----------------------------------------------------------------------------------------
% \begin{rSection}{Summary} 
% Eager IT Engineering student passionate about technology and innovation, seeking Software Engineering Internship to embrace a relentless pursuit of learning.
% \end{rSection}
\begin{rSection}{Education}
    {\bf Bachelor of Engineering (Information Technology)} \hfill {December 2021 - June 2025 } \\
    University of Mumbai — Thadomal Shahani Engineering College, Mumbai  \hfill CGPA: {\bf 8.62 / 10}

    % {\bf Higher Secondary Certification (Science)} \hfill {2019 - 2021} \\
    % D. G. Ruparel College of Arts, Science and Commerce, Mumbai \hfill Percentage: {\bf 94.83 \%}

\end{rSection}



\begin{rSection}{WORK EXPERIENCE}

    \textbf{Full Stack Developer Intern}, \textit{Arxena Inc.} (\href{https://arxena.com}{arxena.com})  \hfill June 2024 - July 2024

    \begin{itemize}
        \itemsep -6pt {}
        \item Improved efficiency of the recruitment process by enhancing the existing
              pipeline and developing a \textbf{GPT-powered WhatsApp chatbot} in
              \textbf{Node.js, NestJS, WebSocket API, GraphQL and React.js}, streamlining
              communication with candidates.
        \item  Managed, Optimized and Deployed applications running on \textbf{AWS EC2},
              reducing costs and improving reliability.
              \item Migrated core and active services between AWS accounts, \textbf{maintaining configuration integrity} and \textbf{minimizing service disruption}.
    \end{itemize}

    \textbf{Information Technology Intern}, \textit{Kraymera Study Abroad} (\href{https://kraymera.com}{kraymera.com})  \hfill June 2023 - September 2023

    \begin{itemize}
        \itemsep -6pt {}
        % \item Implemented the \textbf{CRM systems and processes} using Zoho and developed
        %       \textbf{WhatsApp Business API automation} using the Deluge scripting language,
        %       boosting the lead nurturing process and the conversion rate.
        % \item Installed and configured \textbf{CentOS on a server}, configured
        %       \textbf{networking devices and GSM gateway} for telecommunications leading to
        %       increase in customer acquisition.
              \item Developed and implemented a \textbf{CRM system} with \textbf{automated WhatsApp Business API processes} using Deluge scripting, deploying CentOS server and configuring networking devices and GSM gateway to drive an increase in lead conversion and enhance customer acquisition strategies.
    \end{itemize}
\end{rSection}

%----------------------------------------------------------------------------------------
%	WORK EXPERIENCE SECTION
%----------------------------------------------------------------------------------------

\begin{rSection}{PROJECTS}
    \vspace{-1.25em}

    % \item \textbf{Kisaan Setu}, \textit{React.js, Node.js, Express.js MongoDB, Docker, Redux, Tailwind CSS, Leaflet} | (\href{https://kisaansetu.netlify.app/}{Website}, \href{https://github.com/ninadnaik10/KisaanSetu}{GitHub}) \hfill {February 2024}
    % \begin{itemize}
    %     \setlength\itemsep{-0.6em}
    %     \item Developed \textbf{a full-stack web application} to facilitate farmers and sellers to buy and sell crops, seeds, fertilizers, and equipment, detect and cure crop disease and connnect with other sellers in the region through a map.
    %           % \item Played a key role in \textbf{constructing the database} and \textbf{integrating programming language compilers} on the AWS Cloud.
    %     \item Implemented key features - Online Marketplace, AI-driven Crop Disease Detection and Cure suggestion using Gemini API, Interactive Map using Leaflet and a Complaint Registration portal.
    % \end{itemize}

    % \item \textbf{FireSense}, \textit{Python, YOLOv8, Flask, Flutter, Firebase Cloud Messaging} | (\href{https://github.com/ninadnaik10/FireSense} {GitHub}) \hfill {March 2024}
    % \begin{itemize}
    %     \setlength\itemsep{-0.6em}
    %     \item Developed a \textbf{Computer Vision system} to detect fire at an early stage in
    %           real-time and alert the user through an app.
    %     \item Trained a \textbf{YOLOv8 model} on a custom dataset consisting of around
    %           \textbf{2400 labelled images} in the YOLO format.
    %     \item Created a mobile app in Flutter that displays the captured image as a
    %           notification using FCM.
    %           % \item Achieved \textbf{Precision of 88\%} and \textbf{Recall of 79\%} along with \textbf{87\% Mean Average Precision} at an IoU threshold 0.5.
    % \end{itemize}
 \item \textbf{Shortomega}, \textit{React.js, Next.js, Node.js, NestJS, TypeScript, Redis, Docker} | (\href{https://shortomega.ninadnaik.xyz} {Website}, \href{https://github.com/ninadnaik10/shortomega} {GitHub})  \hfill {October 2024}
    \begin{itemize}
        \setlength\itemsep{-0.6em}
        \item  Developed a \textbf{ Full Stack URL Shortening App} that generates short URLs for long URLs, with a focus on performance.
        \item Implemented \textbf{Rate Limiting, Caching and Analytics} features using Redis, and \textbf{Dockerized} the application for easy deployment. Leveraged Next.js \textbf{Server Side Rendering} for fast webpage load at client side.
    \end{itemize}
    \item \textbf{CodeIt}, \textit{React.js, Flask, Firebase, Docker, Amazon Web Services EC2} | (\href{https://codeitonline.xyz/}{Website}, \href{https://github.com/ninadnaik10/codeit}{GitHub}) \hfill {October 2023}
    \begin{itemize}
        \setlength\itemsep{-0.6em}
        \item Developed \textbf{a full-stack specialized online coding platform} designed for
              instructors to establish groups, post coding challenges, and oversee learners'
              solution submissions. Designed and built an intuitive dashboard for learner and
              instructors, integrated code editor and analysis dashboard.
              % \item Played a key role in \textbf{constructing the database} and \textbf{integrating programming language compilers} on the AWS Cloud.
              % \item Developed a feature enabling instructors to \textbf{create and submit custom coding questions along with test cases}, or select from a library of existing questions. Additionally, implemented a \textbf{dedicated practice portal} for learners.
    \end{itemize}

    \item \textbf{SpeakSure}, \textit{Python, Librosa, scikit-learn, Kivy, MongoDB} | (\href{https://github.com/ninadnaik10/SpeakSure} {GitHub}) \hfill {April 2023}
    \begin{itemize}
        \setlength\itemsep{-0.6em}
        \item  Led a team in the creation of \textbf{an AI Public Speaking Practice app}, with
              a primary focus on analyzing speech metrics such as \textbf{filler words,
                  speech pace, and confidence levels.} Won 3rd prize in PyExpo 2.0 Hackathon.
        \item Played a crucial role in implementing \textbf{speech-to-text transcription and
                  training the AI model using MLPClassifier}.
              % \item Achieved the \textbf{3rd prize} in the Department-level Mini Project Competition (PyExpo 2.0).
    \end{itemize}
    \item \textbf{PassVault}, \textit{Java, Swing, SQLite, Netbeans} (\href{https://github.com/ninadnaik10/PassVault} {GitHub}) \hfill {December 2022}
    \begin{itemize}
        \setlength\itemsep{-0.6em}
        \item Created \textbf{a Password Manager app for desktop} leveraging SHA-256 hashing
              and AES encryption for robust security.
        \item Developed password generation, creation and deletion features. Secured
              \textbf{2nd prize in Javagenix Hackathon}.
    \end{itemize}

\end{rSection}

\begin{rSection}{SKILLS}

    \begin{tabular}{ l @{\hspace{6ex}} l }
        \textbf{Languages}: C, C++, Java, Python, JavaScript     & \textbf{Web}: HTML, CSS, Tailwind CSS, React.js, TypeScript            \\
        \textbf{Frameworks}: Next.js, Node.js, Express.js, NestJS & \textbf{Databases}: MySQL, PostgreSQL, MongoDB,  Cloud Firestore       \\
        \textbf{App}: Flutter                                    & \textbf{Platforms \& Tools}: Linux, Docker, Git, GitHub, AWS, Firebase\end{tabular}
\end{rSection}

\begin{rSection}{ROLES OF RESPONSIBILITY}

    \textbf{Responsible Computing Fellow}, \textit{SimPPL} (\href{https://simppl.org/}{simppl.org}) \hfill March 2024 - June 2024
    \begin{itemize}
        \item Contributed to the front-facing \textbf{kitchen website of SimPPL}
              (\href{https://kitchen.simppl.org/}{link}) displaying information about ongoing
              research in the organisation. Built a dynamic UI for displaying the research
              information pages with \textbf{Next.js and Material UI}.
    \end{itemize}

    \textbf{App Team Member}, \textit{Developers' Club - TSEC} (\href{https://www.linkedin.com/in/developer-s-club-tsec/}{LinkedIn})\hfill March 2023 - March 2024
    % \begin{itemize}
    %     \itemsep -6pt {}
    %     \item \textbf{Facilitated organized sessions} to guide and mentor \textbf{100+ students} in the field of App Development using Flutter.
    %     \item Contributed to the maintenance and enhancement of \textbf{the official app for TSEC} that displays timetable, information, attendance history, sends notifications and alers. It has \textbf{1000+ downloads} on Google Play Store and Apple App Store.
    % \end{itemize}

    \textbf{Core Team Member}, \textit{Google Developer Student Clubs - TSEC} (\href{https://gdsc.community.dev/thadomal-shahani-engineering-college-mumbai/}{Website})  \hfill August 2022 - April 2024
    % \begin{itemize}
    %     \itemsep -6pt {}
    %     \item Conducted \textbf{multiple events and workshops} on Artificial Intelligence, Machine Learning, Cloud Computing, Android Development, Cybersecurity UI/UX Design and organized Mockupp - a Design Hackathon with a total of 794 attendees.
    %     \item \textbf{Interviewed the candidates} seeking to join the Junior Core Team, evaluating their suitability for the role.
    % \end{itemize}

\end{rSection}


%----------------------------------------------------------------------------------------
\begin{rSection}{Achievements \& Extracurricular Activity}
    \begin{itemize}
        \setlength\itemsep{-0.6em}
        \item \textbf{First Runner Up — CSS Battle} among \textbf{100+ participants}, CSI TSEC. \hfill {September 2022}
        \item \textbf{First Runner Up — Javagenix} (Java Project Hackathon) among \textbf{30+ teams}, IT Department, TSEC. \hfill {October 2022}
        \item \textbf{Second Runner Up — PyExpo 2.0} (Python Project Hackathon) among \textbf{30+ teams}, IT Department, TSEC. \hfill {April 2023}
        \item Secured a position in the \textbf{Top 10} among \textbf{53 teams} in ACE Hacks
              1.0, ACE Mumbai. \hfill{March 2024}
        \item Organizer of \textbf{\textit{GeekSpace}} Tech Community
              (\href{https://geekspaceclub.xyz}{Website}) of \textbf{480+ members} helping
              people get into tech.
        \item Solved \textbf{250+} Data Structures and Algorithms problems on
              \textbf{Leetcode, Codeforces} and other platforms.
    \end{itemize}

\end{rSection}

%----------------------------------------------------------------------------------------

\end{document}